% Options for packages loaded elsewhere
\PassOptionsToPackage{unicode}{hyperref}
\PassOptionsToPackage{hyphens}{url}
%
\documentclass[
]{article}
\usepackage{lmodern}
\usepackage{amssymb,amsmath}
\usepackage{ifxetex,ifluatex}
\ifnum 0\ifxetex 1\fi\ifluatex 1\fi=0 % if pdftex
  \usepackage[T1]{fontenc}
  \usepackage[utf8]{inputenc}
  \usepackage{textcomp} % provide euro and other symbols
\else % if luatex or xetex
  \usepackage{unicode-math}
  \defaultfontfeatures{Scale=MatchLowercase}
  \defaultfontfeatures[\rmfamily]{Ligatures=TeX,Scale=1}
\fi
% Use upquote if available, for straight quotes in verbatim environments
\IfFileExists{upquote.sty}{\usepackage{upquote}}{}
\IfFileExists{microtype.sty}{% use microtype if available
  \usepackage[]{microtype}
  \UseMicrotypeSet[protrusion]{basicmath} % disable protrusion for tt fonts
}{}
\makeatletter
\@ifundefined{KOMAClassName}{% if non-KOMA class
  \IfFileExists{parskip.sty}{%
    \usepackage{parskip}
  }{% else
    \setlength{\parindent}{0pt}
    \setlength{\parskip}{6pt plus 2pt minus 1pt}}
}{% if KOMA class
  \KOMAoptions{parskip=half}}
\makeatother
\usepackage{xcolor}
\IfFileExists{xurl.sty}{\usepackage{xurl}}{} % add URL line breaks if available
\IfFileExists{bookmark.sty}{\usepackage{bookmark}}{\usepackage{hyperref}}
\hypersetup{
  pdftitle={Bibliography},
  hidelinks,
  pdfcreator={LaTeX via pandoc}}
\urlstyle{same} % disable monospaced font for URLs
\usepackage[margin=1in]{geometry}
\usepackage{graphicx,grffile}
\makeatletter
\def\maxwidth{\ifdim\Gin@nat@width>\linewidth\linewidth\else\Gin@nat@width\fi}
\def\maxheight{\ifdim\Gin@nat@height>\textheight\textheight\else\Gin@nat@height\fi}
\makeatother
% Scale images if necessary, so that they will not overflow the page
% margins by default, and it is still possible to overwrite the defaults
% using explicit options in \includegraphics[width, height, ...]{}
\setkeys{Gin}{width=\maxwidth,height=\maxheight,keepaspectratio}
% Set default figure placement to htbp
\makeatletter
\def\fps@figure{htbp}
\makeatother
\setlength{\emergencystretch}{3em} % prevent overfull lines
\providecommand{\tightlist}{%
  \setlength{\itemsep}{0pt}\setlength{\parskip}{0pt}}
\setcounter{secnumdepth}{-\maxdimen} % remove section numbering

\title{Bibliography}
\author{}
\date{\vspace{-2.5em}}

\begin{document}
\maketitle

\hypertarget{refs}{}
\leavevmode\hypertarget{ref-RN165}{}%
1. SUPPLEMENTARY information., p. 3

\leavevmode\hypertarget{ref-RN6}{}%
2. Aizen MA, Ezcurra C. 1998. High incidence of plant-animal mutualisms
in the woody flora of the temperate forest of southern south america:
Biogeographical origin and present ecological significance.
\emph{Ecologia Austral}. 8:217--36

\leavevmode\hypertarget{ref-RN8}{}%
3. Aizen MA, Smith‐Ramírez C, Morales CL, Vieli L, Sáez A, et al. 2018.
Coordinated species importation policies are needed to reduce serious
invasions globally: The case of alien bumblebees in south america.
\emph{Journal of Applied Ecology}. 56(1):100--106

\leavevmode\hypertarget{ref-RN7}{}%
4. Aizen MA, Vázquez DP, Smith-Ramírez C. 2002. Historia natural y
conservación de los mutualismos planta-animal del bosque templado de
sudamérica austral. \emph{Revista Chilena de Historia Natural}.
75(1):79--97

\leavevmode\hypertarget{ref-RN9}{}%
5. Amico GC, Vidal-Russell R, Nickrent DL. 2007. Phylogenetic
relationships and ecological speciation in the mistletoe tristerix
(loranthaceae): The influence of pollinators, dispersers, and hosts.
\emph{American Journal of Botany}. 94(4):558--67

\leavevmode\hypertarget{ref-RN10}{}%
6. Anderson GJ, Bernardello G, Lopez PS, Crawford DJ, Stuessy TF. 2000.
Reproductive biology ofWahlenbergia (campanulaceae) endemic to robinson
crusoe island (chile). \emph{Plant Systematics and Evolution}.
223(1-2):109--23

\leavevmode\hypertarget{ref-RN3}{}%
7. Anderson GJ, Bernardello G, Stuessy TF, Crawford DJ. 2001. Breeding
system and pollination of selected plants endemic to juan fernández
islands. \emph{American Journal of Botany}. 88(2):220--33

\leavevmode\hypertarget{ref-RN101}{}%
8. Anic V, Henríquez CA, Abades SR, Bustamante RO. 2015. Number of
conspecifics and reproduction in the invasive plant
\textless i\textgreater Eschscholzia
californica\textless/i\textgreater{} (papaveraceae): Is there a
pollinator-mediated allee effect? \emph{Plant Biology}. 17(3):720--27

\leavevmode\hypertarget{ref-RN12}{}%
9. Araneda Durán X, Breve Ulloa R, Aguilera Carrillo J, Lavín Contreras
J, Toneatti Bastidas M. 2010. Evaluation of yield component traits of
honeybee-pollinated (apis mellifera l.)Rapeseed canola (brassica napus
l.). \emph{Chilean journal of agricultural research}. 70(2):309--14

\leavevmode\hypertarget{ref-RN11}{}%
10. Araneda X, Caniullan R, Catalán C, Martínez M, Morales D, Rodríguez
M. 2015. Nutritional contribution of pollen from species pollinated by
bees (apis mellifera l.) in the araucanía region of chile. \emph{Revista
de la Facultad de Ciencias Agrarias}. 47(1):139--44

\leavevmode\hypertarget{ref-RN13}{}%
11. Arismendi N, Bruna A, Zapata N, Vargas M. 2016. Molecular detection
of the tracheal mite locustacarus buchneri in native and non-native
bumble bees in chile. \emph{Insectes Sociaux}. 63(4):629--33

\leavevmode\hypertarget{ref-RN14}{}%
12. Armesto JJ, Smith-Ramírez C, Carmona MR, Celis-Diez JL, Díaz IA, et
al. 2009. Old-growth temperate rainforests of south america:
Conservation, plant--animal interactions, and baseline biogeochemical
processes. In \emph{Old-growth forests}, ed C Wirth, G Gleixner, M
Heimann, pp. 367--90. Berlin, Heidelberg: Springer Berlin Heidelberg.
pp. ed.

\leavevmode\hypertarget{ref-RN15}{}%
13. Arretz PV, Macfarlane RP. 1986. The introduction of bombus ruderatus
to chile for red clover pollination. \emph{Bee World}. 67(1):15--22

\leavevmode\hypertarget{ref-RN16}{}%
14. Arroyo MK, Armesto JJ, Primak R. 1983. Tendencias altitudinales y
latitudinales en mecanismos de polinización en la zona andina de los
andes templados de sudamérica* *. \emph{Revista Chilena de Historia
Natural}. 56:159--80

\leavevmode\hypertarget{ref-RN167}{}%
15. Arroyo MK, Primack RB, Armesto JJ. 1982. Community studies in
pollination ecology in the high temperate andes of central chile. I.
Pollinaiton mechanisms and individual variation. \emph{American Journal
of Botany}. 69(1):82--97

\leavevmode\hypertarget{ref-RN93}{}%
16. Arroyo MK, Squeo. 1990. Relationships between plant breeding systems
and pollination. In \emph{Biological approaches and evolutionary trends
in plants}, pp. 205--24. London, UK: Academic Press. pp. ed.

\leavevmode\hypertarget{ref-RN166}{}%
17. Arroyo MK, Uslar P. 1993. Breeding systems in a temperate
mediterranean-type climate montane sclerophyllus forest in central
chile. \emph{Botanical Journal of the Linnean Society}. 111:83--102

\leavevmode\hypertarget{ref-RN102}{}%
18. Arroyo MTK, Armesto JJ, Primack RB. 1985. Community studies in
pollination ecology in the high temperate andes of central chile ii.
Effect of temperature on visitation rates and pollination possibilities.
\emph{Plant Systematics and Evolution}. 149(3-4):187--203

\leavevmode\hypertarget{ref-RN106}{}%
19. Arroyo MTK, Dudley LS, Jespersen G, Pacheco DA, Cavieres LA. 2013.
Temperature-driven flower longevity in a high-alpine species of
\textless i\textgreater Oxalis\textless/i\textgreater{} influences
reproductive assurance. \emph{New Phytologist}. 200(4):1260--8

\leavevmode\hypertarget{ref-RN105}{}%
20. Arroyo MTK, Humaña AM, Domínguez D, Jespersen G. 2012. Incomplete
trimorphic incompatibility expression in oxalis compacta gill. Ex hook.
Et arn. Subsp. Compacta in the central chilean andes. \emph{Gayana.
Botánica}. 69(1):88--99

\leavevmode\hypertarget{ref-RN103}{}%
21. Arroyo MTK, Muñoz MS, Henríquez C, Till-Bottraud I, Pérez F. 2006.
Erratic pollination, high selfing levels and their correlates and
consequences in an altitudinally widespread above-tree-line species in
the high andes of chile. \emph{Acta Oecologica}. 30(2):248--57

\leavevmode\hypertarget{ref-RN17}{}%
22. Arroyo MTK, Pacheco DA, Dudley LS. 2017. Functional role of
long-lived flowers in preventing pollen limitation in a high elevation
outcrossing species. \emph{AoB PLANTS}. 9(6):plx050

\leavevmode\hypertarget{ref-RN18}{}%
23. Arroyo MTK, Pérez F, Jara-Arancio P, Pacheco D, Vidal P, Flores MF.
2019. Ovule bet-hedging at high elevation in the south american andes:
Evidence from a phylogenetically controlled multispecies study.
\emph{Journal of Ecology}. 107(2):668--83

\leavevmode\hypertarget{ref-RN19}{}%
24. Belmonte E, Cardemil L, Arroyo MTK. 1994. Floral nectary structure
and nectar composition in eccremocarpus scaber (bignoniaceae), a
hummingbird-pollinated plant of central chile. \emph{American Journal of
Botany}. 81(4):493

\leavevmode\hypertarget{ref-RN21}{}%
25. Bernardello G, Aguilar R, Anderson GJ. 2004. The reproductive
biology of \textless i\textgreater Sophora
fernandeziana\textless/i\textgreater{} (leguminosae), a vulnerable
endemic species from isla robinson crusoe. \emph{American Journal of
Botany}. 91(2):198--206

\leavevmode\hypertarget{ref-RN1}{}%
26. Bernardello G, Anderson GJ, Stuessy TF, Crawford DJ. 2001. A survey
of floral traits, breeding systems, floral visitors, and pollination
systems of the angiosperms of the juan fernández islands (chile).
\emph{The Botanical Review}. 67(3):255--308

\leavevmode\hypertarget{ref-RN20}{}%
27. Bernardello G, Galetto L, Anderson GJ. 2000. Floral nectary
structure and nectar chemical composition of some species from robinson
crusoe island (chile). \emph{Canadian Journal of Botany}. 78:862--72

\leavevmode\hypertarget{ref-RN107}{}%
28. Botto-Mahan C, Ojeda-Camacho M. 2000. THE importance of floral
damage for pollinator visitation in alstroemeria ligtu l.'.
\emph{Revista Chilena de Entomología}. 26:73--76

\leavevmode\hypertarget{ref-RN108}{}%
29. Botto-Mahan C, Pohl N, Medel R. 2004. Nectar guide fluctuating
asymmetry does not relate to female fitness in mimulus luteus.
\emph{Plant Ecology formerly ``Vegetatio''}. 174(2):347--52

\leavevmode\hypertarget{ref-RN109}{}%
30. Botto-Mahan C, Ramírez PA, Gloria Ossa C, Medel R, Ojeda-Camacho M,
González AV. 2011. Floral herbivory affects female reproductive success
and pollinator visitation in the perennial herb
\textless i\textgreater Alstroemeria ligtu\textless/i\textgreater{}
(alstroemeriaceae). \emph{International Journal of Plant Sciences}.
172(9):1130--6

\leavevmode\hypertarget{ref-RN110}{}%
31. Caballero P, Ossa CG, Gonzáles WL, González-Browne C, Astorga G, et
al. 2013. Testing non-additive effects of nectar-robbing ants and
hummingbird pollination on the reproductive success of a parasitic
plant. \emph{Plant Ecology}. 214(4):633--40

\leavevmode\hypertarget{ref-RN22}{}%
32. Candia AB, Medel R, Fontúrbel FE. 2014. Indirect positive effects of
a parasitic plant on host pollination and seed dispersal. \emph{Oikos}.
123(11):1371--6

\leavevmode\hypertarget{ref-RN111}{}%
33. Cares-Suárez R, Poch T, Acevedo RF, Acosta-Bravo I, Pimentel C, et
al. 2011. Do pollinators respond in a dose-dependent manner to flower
herbivory?: An experimental assessment in loasa tricolor (loasaceae).
\emph{Gayana. Botánica}. 68(2):176--81

\leavevmode\hypertarget{ref-RN112}{}%
34. Carvallo G, Medel R. 2005. The modular structure of the floral
phenotype in mimulus luteus var. Luteus (phrymaceae). \emph{Revista
chilena de historia natural}. 78(4):665--72

\leavevmode\hypertarget{ref-RN113}{}%
35. Carvallo GO, Medel R. 2010. Effects of herkogamy and inbreeding on
the mating system of mimulus luteus in the absence of pollinators.
\emph{Evolutionary Ecology}. 24(2):509--22

\leavevmode\hypertarget{ref-RN114}{}%
36. Carvallo GO, Medel R. 2016. Heterospecific pollen transfer from an
exotic plant to native plants: Assessing reproductive consequences in an
andean grassland. \emph{Plant Ecology \& Diversity}. 9(2):151--57

\leavevmode\hypertarget{ref-RN115}{}%
37. Carvallo GO, Medel R, Navarro L. 2013. Assessing the effects of
native plants on the pollination of an exotic herb, the blueweed echium
vulgare (boraginaceae). \emph{Arthropod-Plant Interactions}.
7(5):475--84

\leavevmode\hypertarget{ref-RN168}{}%
38. Cavieres LA, Peñaloza G AP, Arroyo MTK. 1998. Efectos del tamaño
floral y densidad de flores en la visita de insectos polinizadores en
alstroemeria pallida graham (amaryllidaceae). \emph{Gayana Botánica}.
55(1):1--10

\leavevmode\hypertarget{ref-RN116}{}%
39. Celedón-Neghme C, Gonzáles WL, Gianoli E. 2007. Cost and benefits of
attractive floral traits in the annual species madia sativa
(asteraceae). \emph{Evolutionary Ecology}. 21(2):247--57

\leavevmode\hypertarget{ref-RN23}{}%
40. Cepeda-Pizarro J, Pola P M, González CR. 2015. Efecto de la fase
fenológica de verano sobre algunas características del ensamble de
diptera registrado en una vega altoandina del desierto transicional de
chile. \emph{Idesia (Arica)}. 33(1):49--58

\leavevmode\hypertarget{ref-RN24}{}%
41. Chalcoff VR, Aizen MA, Ezcurra C. 2012. Erosion of a pollination
mutualism along an environmental gradient in a south andean treelet,
embothrium coccineum (proteaceae). \emph{Oikos}. 121(3):471--80

\leavevmode\hypertarget{ref-RN25}{}%
42. Chetelat RT, Pertuzé RA, Faúndez L, Graham EB, Jones CM. 2009.
Distribution, ecology and reproductive biology of wild tomatoes and
related nightshades from the atacama desert region of northern chile.
\emph{Euphytica}. 167(1):77--93

\leavevmode\hypertarget{ref-RN26}{}%
43. Cisterna J, Murúa M. 2018. Contrasting floral morphology and
breeding systems in two subspecies of calceolaria corymbosa in central
chile. \emph{Gayana. Botánica}. 75(1):544--48

\leavevmode\hypertarget{ref-RN117}{}%
44. Cooley AM, Carvallo G, Willis JH. 2008. Is floral diversification
associated with pollinator divergence? Flower shape, flower colour and
pollinator preference in chilean mimulus. \emph{Annals of Botany}.
101(5):641--50

\leavevmode\hypertarget{ref-RN118}{}%
45. Cuartas-Domínguez M, Medel R. 2010. Pollinator-mediated selection
and experimental manipulation of the flower phenotype in chloraea
bletioides: Pollinator-mediated selection in chloraea bletioides.
\emph{Functional Ecology}. 24(6):1219--27

\leavevmode\hypertarget{ref-RN27}{}%
46. Cuartas-Domínguez M, Rojas-Céspedes A, Jara-Arancio P, Arroyo MTK.
2017. Sistema reproductivo de trichopetalum plumosum (ruiz \&amp; pav.)
j.f. Macbr. (Asparagaceae), geófita endémica de chile. \emph{Gayana.
Botánica}. 74(1):73--81

\leavevmode\hypertarget{ref-RN170}{}%
47. DeUgarte PM. 1991. \emph{Agentes polinizantes y rendimiento en el
cultivo de alforfón (fagopyrum esculentum) / pollinating agents and crop
yield of buckwheat (fagopyrum esculentum)}. Thesis thesis. pp.

\leavevmode\hypertarget{ref-RN28}{}%
48. Díaz IA, Armesto JJ, Reid S, Sieving KE, Willson MF. 2005. Linking
forest structure and composition: Avian diversity in successional
forests of chiloé island, chile. \emph{Biological Conservation}.
123(1):91--101

\leavevmode\hypertarget{ref-RN29}{}%
49. Díaz-Forestier J, Gómez M, Celis-Diez JL, Montenegro G. 2016.
Nectary structure in four melliferous plant species native to chile.
\emph{Flora}. 221:100--106

\leavevmode\hypertarget{ref-RN30}{}%
50. Espíndola A, Pliscoff P. 2019. The relationship between pollinator
visits and climatic suitabilities in specialized pollination
interactions. \emph{Annals of the Entomological Society of America}.
112(3):150--57

\leavevmode\hypertarget{ref-RN119}{}%
51. Espinoza CL, Murúa M, Bustamante RO, Marín VH, Medel R. 2012.
Reproductive consequences of flower damage in two contrasting habitats:
The case of viola portalesia (violaceae) in chile. \emph{Revista chilena
de historia natural}. 85(4):503--11

\leavevmode\hypertarget{ref-RN86}{}%
52. Estay P P. 2007. \emph{Bombus en chile: Especies, biología y
manejo}. Santiago, Chile: Instituto de Investigaciones Agropecuarias
(INIA), Centro Regional de Investigación La Platina, Ministerio de
Agricultura. pp. ed.

\leavevmode\hypertarget{ref-RN172}{}%
53. Estay P P, Carvajal M. Domesticación y crianza continua de los
abejorros bombus terrestris y bombus dahlbomii en chile

\leavevmode\hypertarget{ref-RN171}{}%
54. Estay P P, al., al. 2001. Producción de abejorros en chile /
bumblebee production in chile. \emph{Tierra Adentro}.
37(marzo-abril):14--15

\leavevmode\hypertarget{ref-RN31}{}%
55. Estay P P, Wagner V A, Escaff G M. 2001. Evaluación de bombus
dahlbomii (guér.) como agente polinizador de flores de tomate
(lycopersicon esculentum (mill)), bajo condiciones de invernadero.
\emph{Agricultura Técnica}. 61(2):113--19

\leavevmode\hypertarget{ref-RN120}{}%
56. Esterio G, Cares-Suárez R, González-Browne C, Salinas P, Carvallo G,
Medel R. 2013. Assessing the impact of the invasive buff-tailed
bumblebee (bombus terrestris) on the pollination of the native chilean
herb mimulus luteus. \emph{Arthropod-Plant Interactions}. 7(4):467--74

\leavevmode\hypertarget{ref-RN33}{}%
57. Fao F, Agriculture Organization of the United N. 2016. Línea base
del servicio ecosistémico de la polinización en chile: Documento de
síntesis. \emph{I6663ES/1/12.16}

\leavevmode\hypertarget{ref-RN32}{}%
58. Fao F, Agriculture Organization of the United N. 2017. Estado del
arte del servicio ecosistémico de la polinización en chile-paraguay y
peru. FAO

\leavevmode\hypertarget{ref-RN34}{}%
59. Farji-Brener AG, Corley JC. 1998. Successful invasions of
hymenopteran insects into nw patagonia. \emph{Ecologia Austral}.
8:237--49

\leavevmode\hypertarget{ref-RN35}{}%
60. Fontúrbel FE, Jordano P, Medel R. 2015. Scale-dependent responses of
pollination and seed dispersal mutualisms in a habitat transformation
scenario. \emph{Journal of Ecology}. 103(5):1334--43

\leavevmode\hypertarget{ref-RN36}{}%
61. Fontúrbel FE, Jordano P, Medel R. 2017. Plant-animal mutualism
effectiveness in native and transformed habitats: Assessing the coupled
outcomes of pollination and seed dispersal. \emph{Perspectives in Plant
Ecology, Evolution and Systematics}. 28:87--95

\leavevmode\hypertarget{ref-RN37}{}%
62. Fontúrbel FE, Salazar DA, Medel R. 2017. Increased resource
availability prevents the disruption of key ecological interactions in
disturbed habitats. \emph{Ecosphere}. 8(4):e01768

\leavevmode\hypertarget{ref-RN121}{}%
63. Galetto L, Bernardello G, Sosa CA. 1998. The relationship between
floral nectar composition and visitors in lycium (solanaceae) from
argentina and chile: What does it reflect? \emph{Flora}. 193(3):303--14

\leavevmode\hypertarget{ref-RN38}{}%
64. Gómez P P, Lillo D, González AV. 2012. Pollination and breeding
system in adesmia bijuga phil. (Fabaceae), a critically endangered
species in central chile. \emph{Gayana Botánica}. 69(2):286--95

\leavevmode\hypertarget{ref-RN39}{}%
65. González AV, González-Browne C, Salinas P, Murúa M. 2019. Is there
spatial variation in phenotypic selection on floral traits in a
generalist plant--pollinator system? \emph{Evolutionary Ecology}.
33(5):687--700

\leavevmode\hypertarget{ref-RN123}{}%
66. González AV, Murúa M, Ramírez PA. 2014. Temporal and spatial
variation of the pollinator assemblages in alstroemeria ligtu
(alstroemeriaceae). \emph{Revista Chilena de Historia Natural}. 87(5):

\leavevmode\hypertarget{ref-RN124}{}%
67. González AV, Murúa MM, Pérez F. 2015. Floral integration and
pollinator diversity in the generalized plant-pollinator system of
alstroemeria ligtu (alstroemeriaceae). \emph{Evolutionary Ecology}.
29(1):63--75

\leavevmode\hypertarget{ref-RN122}{}%
68. González AV, Pérez F. 2010. Pollen limitation and reproductive
assurance in the flora of the coastal atacama desert.
\emph{International Journal of Plant Sciences}. 171(6):607--14

\leavevmode\hypertarget{ref-RN40}{}%
69. González-Gómez PL, Estades CF. 2009. Is natural selection promoting
sexual dimorphism in the green-backed firecrown hummingbird (sephanoides
sephaniodes)? \emph{Journal of Ornithology}. 150(2):351--56

\leavevmode\hypertarget{ref-RN125}{}%
70. Gonzalez-Gomez PL, Valdivia CE. 2005. Direct and indirect effects of
nectar robbing on the pollinating behavior of patagona gigas
(trochilidae)1. \emph{Biotropica}. 37(4):693--96

\leavevmode\hypertarget{ref-RN41}{}%
71. González-Vaquero RA, Galvani GL. 2016. Antennal sensilla analyses as
useful tools in the revision of the sweat-bee subgenus corynura
(callistochlora) michener (hymenoptera: Halictidae). \emph{Zoologischer
Anzeiger - A Journal of Comparative Zoology}. 262:29--42

\leavevmode\hypertarget{ref-RN42}{}%
72. González-Vaquero RA, Roig-Alsina A, Packer L. 2016. DNA barcoding as
a useful tool in the systematic study of wild bees of the tribe
augochlorini (hymenoptera: Halictidae). \emph{Genome}. 59(10):889--98

\leavevmode\hypertarget{ref-RN44}{}%
73. Guerrero PC, Antinao CA, Vergara-Meriño B, Villagra CA, Carvallo GO.
2019. Bees may drive the reproduction of four sympatric cacti in a
vanishing coastal mediterranean-type ecosystem. \emph{PeerJ}. 7:e7865

\leavevmode\hypertarget{ref-RN43}{}%
74. Guerrero PC, Carvallo GO, Nassar JM, Rojas-Sandoval J, Sanz V, Medel
R. 2012. Ecology and evolution of negative and positive interactions in
cactaceae: Lessons and pending tasks. \emph{Plant Ecology \& Diversity}.
5(2):205--15

\leavevmode\hypertarget{ref-RN126}{}%
75. Humaña AM, Cisternas MA, Valdivia CE. 2008. Breeding system and
pollination of selected orchids of the genus chloraea (orchidaceae) from
central chile. \emph{Flora - Morphology, Distribution, Functional
Ecology of Plants}. 203(6):469--73

\leavevmode\hypertarget{ref-RN87}{}%
76. Humaña AM, Riveros M. 1994. Biología de la reproducción en la
especie trepadora lapageria rosea r. Et p. (Phileiaceae). \emph{Gayana
Botánica}. 51(2):49--55

\leavevmode\hypertarget{ref-RN104}{}%
77. K. Arroyo MT, Till-Bottraud I, Torres C, Henríquez CA, Martínez J.
2007. Display size preferences and foraging habits of high andean
butterflies pollinating \textless i\textgreater Chaetanthera
lycopodioides\textless/i\textgreater{} (asteraceae) in the subnival of
the central chilean andes. \emph{Arctic, Antarctic, and Alpine
Research}. 39(3):347--52

\leavevmode\hypertarget{ref-RN127}{}%
78. Ladd PG, Arroyo MTK. 2009. Comparisons of breeding systems between
two sympatric species, \textless i\textgreater Nastanthus
spathulatus\textless/i\textgreater{} (calyceraceae) and
\textless i\textgreater Rhodophiala rhodolirion\textless/i\textgreater{}
(amaryllidaceae), in the high andes of central chile. \emph{Plant
Species Biology}. 24(1):2--10

\leavevmode\hypertarget{ref-RN130}{}%
79. Lander TA, Bebber DP, Choy CTL, Harris SA, Boshier DH. 2011. The
circe principle explains how resource-rich land can waylay pollinators
in fragmented landscapes. \emph{Current Biology}. 21(15):1302--7

\leavevmode\hypertarget{ref-RN129}{}%
80. Lander TA, Boshier DH, Harris SA. 2010. Fragmented but not isolated:
Contribution of single trees, small patches and long-distance pollen
flow to genetic connectivity for gomortega keule, an endangered chilean
tree. \emph{Biological Conservation}. 143(11):2583--90

\leavevmode\hypertarget{ref-RN128}{}%
81. Lander TA, Harris SA, Boshier DH. 2009. Flower and fruit production
and insect pollination of the endangered chilean tree, gomortega keule
in native forest, exotic pine plantation and agricultural environments.
\emph{Revista chilena de historia natural}. 82(3):403--12

\leavevmode\hypertarget{ref-RN2}{}%
82. Lehnebach C, Riveros M. 2003. Pollination biology of the chilean
endemic orchid chloraea lamellata. \emph{Biodiversity \& Conservation}.
12(8):1741--51

\leavevmode\hypertarget{ref-RN131}{}%
83. Lemaitre AB, Pinto CF, Niemeyer HM. 2014. Generalized pollination
system: Are floral traits adapted to different pollinators?
\emph{Arthropod-Plant Interactions}. 8(4):261--71

\leavevmode\hypertarget{ref-RN45}{}%
84. López-Sepúlveda P, Takayama K, Crawford DJ, Greimler J, Peñailillo
P, et al. 2017. Biogeography and genetic consequences of anagenetic
speciation of \textless i\textgreater R\textless/i\textgreater{}
\textless i\textgreater haphithamnus venustus\textless/i\textgreater{}
(verbenaceae) in the juan fernández archipelago, chile: Insights from
aflp and ssr markers: GENETIC diversity and speciation in
\textless i\textgreater RHAPHITHAMNUS\textless/i\textgreater.
\emph{Plant Species Biology}. 32(3):223--37

\leavevmode\hypertarget{ref-RN46}{}%
85. Magrach A, Larrinaga AR, Santamaría L. 2012. Effects of matrix
characteristics and interpatch distance on functional connectivity in
fragmented temperate rainforests: \textless i\textgreater Matrix and
distance affect functional connectivity\textless/i\textgreater.
\emph{Conservation Biology}. 26(2):238--47

\leavevmode\hypertarget{ref-RN47}{}%
86. Mansur L, Gonzalez M, Rojas I, Salas P. 2004. Self-incompatibility
in the chilean endemic genus leucocoryne lindley. \emph{Journal of the
American Society for Horticultural Science}. 129(6):836--38

\leavevmode\hypertarget{ref-RN132}{}%
87. Marco DE, Arroyo MTK. 1998. The breeding system of
\textless i\textgreater Oxalis squamata\textless/i\textgreater{} , a
tristylous south american species. \emph{Botanica Acta}. 111(6):497--504

\leavevmode\hypertarget{ref-RN48}{}%
88. Mathiasen P, Rovere AE, Premoli AC. 2007. Genetic structure and
early effects of inbreeding in fragmented temperate forests of a
self-incompatible tree, embothrium coccineum. \emph{Conservation
Biology}. 21(1):232--40

\leavevmode\hypertarget{ref-RN49}{}%
89. Medan D, Arce ME. 1999. Reproductive biology of the andean-disjunct
genus retanilla (rhamnaceae). \emph{Plant Systematics and Evolution}.
218(3-4):281--98

\leavevmode\hypertarget{ref-RN50}{}%
90. Medan D, D'Ambrogio AC. 1998. Reproductive biology of the
andromonoecious shrub trevoa quinquenervia (rhamnaceae). \emph{Botanical
Journal of the Linnean Society}. 126(3):191--206

\leavevmode\hypertarget{ref-RN133}{}%
91. Medan D, Montaldo NH. 2005. Ornithophily in the rhamnaceae: The
pollination of the chilean endemic colletia ulicina. \emph{Flora -
Morphology, Distribution, Functional Ecology of Plants}. 200(4):339--44

\leavevmode\hypertarget{ref-RN134}{}%
92. Medel R, Botto-Mahan C, Kalin-Arroyo M. 2003. POLLINATOR-mediated
selection on the nectar guide phenotype in the andean monkey flower,
mimulus luteus. \emph{Ecology}. 84(7):1721--32

\leavevmode\hypertarget{ref-RN51}{}%
93. Medel R, González-Browne C, Fontúrbel FE. 2018. Pollination in the
chilean mediterranean-type ecosystem: A review of current advances and
pending tasks. \emph{Plant Biology}. 20:89--99

\leavevmode\hypertarget{ref-RN135}{}%
94. Medel R, Valiente A, Botto-Mahan C, Carvallo G, Pérez F, et al.
2007. The influence of insects and hummingbirds on the geographical
variation of the flower phenotype in mimulus luteus. \emph{Ecography}.
30(6):812--18

\leavevmode\hypertarget{ref-RN52}{}%
95. Miranda Villalón JMM. 2002. \emph{Determinación de la viabilidad en
la carga polínica de insectos, que visitan flores de avellano chileno
(gevuina avellana mol.)}. Thesis thesis. pp.

\leavevmode\hypertarget{ref-RN100}{}%
96. Moldenke AL, Moldenke HN. 1979. Pollination ecology as an assay for
ecosystemic organization: Convergent evolution in chile and california.
\emph{Phytologia}. 42(1):415--54

\leavevmode\hypertarget{ref-RN94}{}%
97. Moldenke AR. 1976. Evolutionary history and diversity of the bee
faunas of chile and pacific north america. \emph{Wasmann Journal of
Biology}. 3(2):147--78

\leavevmode\hypertarget{ref-RN137}{}%
98. Molina-Montenegro MA, Badano EI, Cavieres LA. 2008. Positive
interactions among plant species for pollinator service: Assessing the
``magnet species'' concept with invasive species. \emph{Oikos}.
117(12):1833--9

\leavevmode\hypertarget{ref-RN136}{}%
99. Molina-Montenegro MA, Cavieres LA. 2006. EFFECT of density and
flower size on the reproductive success of nothoscordum graminum
(alliaceae). \emph{Gayana. Botánica}. 63(1):93--98

\leavevmode\hypertarget{ref-RN173}{}%
100. Montalva, al. 2011. Status report on the chilean bumblebee, bombus
dahlbomii. Ministry of Environment Chile (MMA)

\leavevmode\hypertarget{ref-RN53}{}%
101. Montalva J, Dudley L, Arroyo MK, Retamales H, Abrahamovich AH.
2011. Geographic distribution and associated flora of native and
introduced bumble bees (
\textless i\textgreater Bombus\textless/i\textgreater{} spp.) in chile.
\emph{Journal of Apicultural Research}. 50(1):11--21

\leavevmode\hypertarget{ref-RN54}{}%
102. Montalva J, Sepulveda V, Vivallo F, Silva DP. 2017. New records of
an invasive bumble bee in northern chile: Expansion of its range or new
introduction events? \emph{Journal of Insect Conservation}.
21(4):657--66

\leavevmode\hypertarget{ref-RN88}{}%
103. Monzón V. 2015. \emph{Guía de abejas nativas de la región del
maule}. Talca, Chile: Universidad Católica del Maule. pp. ed.

\leavevmode\hypertarget{ref-RN138}{}%
104. Muñoz AA, Arroyo MTK. 2004. Negative impacts of a vertebrate
predator on insect pollinator visitation and seed output in chuquiraga
oppositifolia, a high andean shrub. \emph{Oecologia}. 138(1):66--73

\leavevmode\hypertarget{ref-RN139}{}%
105. Muñoz AA, Arroyo MTK. 2006. Pollen limitation and spatial variation
of reproductive success in the insect-pollinated shrub chuquiraga
oppositifolia (asteraceae) in the chilean andes. \emph{Arctic,
Antarctic, and Alpine Research}. 38(4):608--13

\leavevmode\hypertarget{ref-RN140}{}%
106. Muñoz AA, Cavieres LA. 2008. The presence of a showy invasive plant
disrupts pollinator service and reproductive output in native alpine
species only at high densities: Invasive impacts on native species
pollination. \emph{Journal of Ecology}. 96(3):459--67

\leavevmode\hypertarget{ref-RN4}{}%
107. Muñoz AA, Cavieres LA. 2019. Sharing of pollinators between the
invasive taraxacum officinale and co-flowering natives is not related to
floral similarity in the high-andes. \emph{Flora}. 261:151491

\leavevmode\hypertarget{ref-RN141}{}%
108. Muñoz AA, Celedon-Neghme C, Cavieres LA, Arroyo MTK. 2005.
Bottom-up effects of nutrient availability on flower production,
pollinator visitation, and seed output in a high-andean shrub.
\emph{Oecologia}. 143(1):126--35

\leavevmode\hypertarget{ref-RN144}{}%
109. Murúa M, Cisterna J, Rosende B. 2014. Pollination ecology and
breeding system of two calceolaria species in chile. \emph{Revista
Chilena de Historia Natural}. 87(1):

\leavevmode\hypertarget{ref-RN55}{}%
110. Murúa M, Espíndola A. 2015. Pollination syndromes in a specialised
plant-pollinator interaction: Does floral morphology predict pollinators
in \textless i\textgreater Calceolaria\textless/i\textgreater{} ?
\emph{Plant Biology}. 17(2):551--57

\leavevmode\hypertarget{ref-RN142}{}%
111. Murúa M, Espíndola A. 2015. Pollination syndromes in a specialised
plant-pollinator interaction: Does floral morphology predict pollinators
in \textless i\textgreater Calceolaria\textless/i\textgreater{} ?
\emph{Plant Biology}. 17(2):551--57

\leavevmode\hypertarget{ref-RN57}{}%
112. Murúa M, Espíndola A, González A, Medel R. 2017. Pollinators and
crossability as reproductive isolation barriers in two sympatric
oil-rewarding calceolaria (calceolariaceae) species. \emph{Evolutionary
Ecology}. 31(4):421--34

\leavevmode\hypertarget{ref-RN143}{}%
113. Murúa M, Espinoza C, Bustamante R, Marín VH, Medel R. 2010. Does
human-induced habitat transformation modify pollinator-mediated
selection? A case study in viola portalesia (violaceae).
\emph{Oecologia}. 163(1):153--62

\leavevmode\hypertarget{ref-RN58}{}%
114. Murúa M, Ramírez MJ, González A. 2019. Is the same pollinator
species equally effective in different populations of the generalist
herb alstroemeria ligtu var. Simsii? \emph{Gayana Botánica}.
76(1):109--14

\leavevmode\hypertarget{ref-RN56}{}%
115. Murúa MM, Grez AA, Simonetti JA. 2011. Changes in wing length in
the pollinator bombus dahlbomii occurring with the fragmentation of the
maulino forest, chile. \emph{Ciencia e investigación agraria}.
38(3):391--96

\leavevmode\hypertarget{ref-RN95}{}%
116. Neira MA, Viscarra RC, Riveros M. 2000. Rubus idaeus l.,
pollinators in x region (chile). \emph{Phyton-International Journal of
Experimental Botany}. 67:43--51

\leavevmode\hypertarget{ref-RN89}{}%
117. Ortiz-Crespo FI. 1986. Consideraciones sobre las migraciones de dos
picaflores neotropicales. \emph{El Honero}. 012(4):298--300

\leavevmode\hypertarget{ref-RN145}{}%
118. Ossa CG, Medel R. 2011. Notes on the floral biology and pollination
syndrome of echinopsis chiloensis (colla) friedrich \& g.d.rowley
(cactaceae) in a population of semiarid chile. \emph{Gayana. Botánica}.
68(2):213--19

\leavevmode\hypertarget{ref-RN59}{}%
119. Pantoja G, Gómez M, Servicio Nacional de A, Contreras C, Grimau L,
et al. 2017. Determination of suitable zones for apitourism using
multi-criteria evaluation in geographic information systems: A case
study in the o'Higgins region, chile. \emph{Ciencia e investigación
agraria}. 44(2):139--53

\leavevmode\hypertarget{ref-RN169}{}%
120. Peralta I, Rodríguez GJ, Arroyo MK. 1992. Breeding system and
aspects of pollination in acacia caven (mol.) mol. (Leguminosae:
Mimosidae) in the mediterranean-type climate zone of central chile.
\emph{Botanische Jahrbücher fur Systematik}. 114(3):297--314

\leavevmode\hypertarget{ref-RN149}{}%
121. Pérez F. 2011. Discordant patterns of morphological and genetic
divergence in the closely related species schizanthus hookeri and s.
Grahamii (solanaceae). \emph{Plant Systematics and Evolution}.
293(1-4):197--205

\leavevmode\hypertarget{ref-RN148}{}%
122. Perez F, Arroyo MTK, Armesto JJ. 2009. Evolution of autonomous
selfing accompanies increased specialization in the pollination system
of schizanthus (solanaceae). \emph{American Journal of Botany}.
96(6):1168--76

\leavevmode\hypertarget{ref-RN147}{}%
123. Pérez F, Arroyo MTK, Medel R. 2007. Phylogenetic analysis of floral
integration in
\textless i\textgreater Schizanthus\textless/i\textgreater{}
(solanaceae): Does pollination truly integrate corolla traits?:
Evolution of floral integration in
\textless i\textgreater Schizanthus\textless/i\textgreater.
\emph{Journal of Evolutionary Biology}. 20(5):1730--8

\leavevmode\hypertarget{ref-RN146}{}%
124. Pérez F, Arroyo MTK, Medel R, Hershkovitz MA. 2006. Ancestral
reconstruction of flower morphology and pollination systems in
\textless i\textgreater Schizanthus\textless/i\textgreater{}
(solanaceae). \emph{American Journal of Botany}. 93(7):1029--38

\leavevmode\hypertarget{ref-RN150}{}%
125. Pérez F, León C, Muñoz T. 2013. How variable is delayed selfing in
a fluctuating pollinator environment? A comparison between a delayed
selfing and a pollinator-dependent schizanthus species of the high
andes. \emph{Evolutionary Ecology}. 27(5):911--22

\leavevmode\hypertarget{ref-RN151}{}%
126. Pohl N, Carvallo G, Botto-Mahan C, Medel R. 2006. Nonadditive
effects of flower damage and hummingbird pollination on the fecundity of
mimulus luteus. \emph{Oecologia}. 149(4):648--55

\leavevmode\hypertarget{ref-RN60}{}%
127. Polidori C, Nieves-Aldrey JL. 2015. Comparative flight morphology
in queens of invasive and native patagonian bumblebees (hymenoptera:
Bombus). \emph{Comptes Rendus Biologies}. 338(2):126--33

\leavevmode\hypertarget{ref-RN61}{}%
128. Ramos-Jiliberto R, Albornoz AA, Valdovinos FS, Smith-Ramírez C,
Arim M, et al. 2009. A network analysis of plant--pollinator
interactions in temperate rain forests of chiloé island, chile.
\emph{Oecologia}. 160(4):697--706

\leavevmode\hypertarget{ref-RN152}{}%
129. Ramos-Jiliberto R, Domínguez D, Espinoza C, López G, Valdovinos FS,
et al. 2010. Topological change of andean plant--pollinator networks
along an altitudinal gradient. \emph{Ecological Complexity}. 7(1):86--90

\leavevmode\hypertarget{ref-RN62}{}%
130. Rebolledo R R, Martínez P H, Palma M R, Aguilera P A, Klein K C.
2004. Actividad de visita de bombus dahlbomi (guérin)y bombus ruderatus
(f.) (hymenoptera:Apidae) sobre trébol rosado (trifolium pratense l.) en
la ix región de la la araucanía, chile. \emph{Agricultura Técnica}.
64(3):

\leavevmode\hypertarget{ref-RN63}{}%
131. Rivera-Hutinel A, Acevedo-Orellana F. 2017. Biología floral y
reproductiva de escallonia pulverulenta (ruiz et pav.) pers.
(Escalloniaceae) y su relación con los visitantes florales. \emph{Gayana
Botánica}. 74(1):82--93

\leavevmode\hypertarget{ref-RN153}{}%
132. Rivera-Hutinel A, Bustamante RO, Marín VH, Medel R. 2012. Effects
of sampling completeness on the structure of plant--pollinator networks.
\emph{Ecology}. 93(7):1593--1603

\leavevmode\hypertarget{ref-RN64}{}%
133. Rovere AE, Smith-Ramírez C, Armesto JJ, Premoli AC. 2006. Breeding
system of embothrium coccineum (proteaceae) in two populations on
different slopes of the andes. \emph{Revista chilena de historia
natural}. 79(2):

\leavevmode\hypertarget{ref-RN91}{}%
134. Rozzi R, Armesto JJ, Correa A, Torres-Mura JC, Salaberry M. 1996.
Avifauna de bosques primarios templados en islas deshabitadas del
archipiélago de chiloé, chile. \emph{Revista Chilena de Historia
Natural}. 69:125--39

\leavevmode\hypertarget{ref-RN154}{}%
135. Rozzi R, Arroyo MK, Armesto JJ. 1997. Ecological factors affecting
gene flow between populations of anarthrophyllum cumingii (papilionaceae)
growing on equatorial- and polar-facing slopes in the andes of central
chile. \emph{Plant Ecology}. 132:171--79

\leavevmode\hypertarget{ref-RN90}{}%
136. Rozzi R, Martínez D, Wilson MF, Sabag C. 1995. Avifauna de los
bosques templados de sudamérica. In \emph{Ecología de los bosques
nativos de chile}, pp. 135--52. Santiago, Chile: Universidad de Chile.
pp. ed.

\leavevmode\hypertarget{ref-RN65}{}%
137. Ruz L. 2002. Bee pollinators introduced to chile: A review. In
\emph{Pollinating bees - the conservation link between agriculture and
nature}, pp. 155--67. Brasilia, Brazil: Ministry of Environment. pp. ed.

\leavevmode\hypertarget{ref-RN155}{}%
138. Ruz L, Herrera R. 2001. PRELIMINARY observations on foraging
activities of bombus dahlbomii and bombus terrestris (hym: APIDAE) on
native and non-native vegetation in chile. \emph{Acta Horticulturae},
pp. 165--69

\leavevmode\hypertarget{ref-RN66}{}%
139. Sanzana M-J, Parra LE, Benítez HA, Espejo J. 2012. Entomofauna
polinizadora de eucalyptus nitens en huertos semilleros del centro sur
de chile. \emph{Bosque}. 33(1):25--31

\leavevmode\hypertarget{ref-RN67}{}%
140. Schmid-Hempel R, Eckhardt M, Goulson D, Heinzmann D, Lange C, et
al. 2014. The invasion of southern south america by imported bumblebees
and associated parasites. \emph{Journal of Animal Ecology}.
83(4):823--37

\leavevmode\hypertarget{ref-RN96}{}%
141. Seguel I, Riveros M, Lehnebach C, Torres A. 1999. Reproductive and
phenological development of ugni molinae turcz. (Myrtaceae).
\emph{Phyton-International Journal of Experimental Botany}. 65:13--21

\leavevmode\hypertarget{ref-RN92}{}%
142. Skewes JC. 2019. La regeneración del bosque templado, los saberes
en tensión y el reconocimiento de la ciudadanía ambiental a partir de la
experiencia de los apicultores. \emph{Desenvolvimento e Meio Ambiente}.
50:73--92

\leavevmode\hypertarget{ref-RN72}{}%
143. Smith-Ramirez C. 1993. Los picaflores y su recurso floral en el
bosque templado de la isla de chiloé, chile. \emph{Revista Chilena de
Historia Natural}. 66:65--73

\leavevmode\hypertarget{ref-RN69}{}%
144. Smith-Ramirez C, Armesto JJ. 1994. Flowering and fruiting patterns
in the temperate rainforest of chiloe, chile--ecologies and climatic
constraints. \emph{The Journal of Ecology}. 82(2):353--65

\leavevmode\hypertarget{ref-RN70}{}%
145. Smith-Ramirez C, Armesto JJ. 1998. Nectarivoría y polinización por
aves en embothrium coccineum (proteaceae) en el bosque templado del sur
de chile. \emph{Revista Chilena de Historia Natural}. 71:51--63

\leavevmode\hypertarget{ref-RN71}{}%
146. Smith-Ramirez C, Armesto JJ. 2003. Foraging behaviour of bird
pollinators on embothrium coccineum (proteaceae) trees in forest
fragments and pastures in southern chile. \emph{Austral Ecology}.
28(1):53--60

\leavevmode\hypertarget{ref-RN98}{}%
147. Smith-Ramirez C, Armesto JJ, Valdovinos C. 2005. \emph{Historia,
biodiversidad y ecología de los bosques costeros de chile}. Santiago,
Chile: Editorial Universitaria. pp. First Edition ed.

\leavevmode\hypertarget{ref-RN74}{}%
148. Smith-Ramírez C, Martinez P, Nuñez M, González C, Armesto JJ. 2005.
Diversity, flower visitation frequency and generalism of pollinators in
temperate rain forests of chiloé island, chile. \emph{Botanical Journal
of the Linnean Society}. 147(4):399--416

\leavevmode\hypertarget{ref-RN75}{}%
149. Smith-Ramírez C, Ramos-Jiliberto R, Valdovinos FS, Martínez P,
Castillo JA, Armesto JJ. 2014. Decadal trends in the pollinator
assemblage of eucryphia cordifolia in chilean rainforests.
\emph{Oecologia}. 176(1):157--69

\leavevmode\hypertarget{ref-RN97}{}%
150. Smith-Ramirez C, Rovere AE, Núñez-Ávila MC, Armesto JJ. 2007.
Habitat fragmentation and reproductive ecology of embothrium coccineum ,
eucryphia cordifolia, and aextoxicon punctatum in southern temperate
rainforests. In \emph{Biodiversity loss and conservation in fragmented
forest landscapes}, pp. 102--19. pp. ed.

\leavevmode\hypertarget{ref-RN68}{}%
151. Smith-Ramírez C, Vieli L, Barahona-Segovia RM, Montalva J,
Cianferoni F, et al. 2018. Las razones de por qué chile debe detener la
importación del abejorro comercial bombus terrestris (linnaeus) y
comenzar a controlarlo. \emph{Gayana}. 8(2):118--27

\leavevmode\hypertarget{ref-RN73}{}%
152. Smith-Ramırez C, Armesto JJ, Figueroa J. 1998. Flowering, fruiting
and seed germination in chilean rain forest myrtaceae: Ecological and
phylogenetic constraints. \emph{Plant Ecology}. 136:119--31

\leavevmode\hypertarget{ref-RN99}{}%
153. Squeo FA. A comparisson of biotic pollination in 2 mountain
transects t the 50-degrees-s, patagonia, chile

\leavevmode\hypertarget{ref-RN156}{}%
154. Suárez LH, Gonzáles WL, Gianoli E. 2009. Foliar damage modifies
floral attractiveness to pollinators in alstroemeria exerens.
\emph{Evolutionary Ecology}. 23(4):545--55

\leavevmode\hypertarget{ref-RN76}{}%
155. Suárez LH, González WL, Gianoli E. 2004. Biología reproductiva de
convolvulus chilensis (convolvulaceae) en una población de aucó
(centro-norte de chile). \emph{Revista Chilena de Historia Natural}.
77(4):

\leavevmode\hypertarget{ref-RN157}{}%
156. Suárez LH, Pérez F, Armesto JJ. 2011. Strong phenotypic variation
in floral design and display traits of an annual tarweed in relation to
small-scale topographic heterogeneity in semiarid chile.
\emph{International Journal of Plant Sciences}. 172(8):1012--25

\leavevmode\hypertarget{ref-RN77}{}%
157. Sun BY, Stussey TF, Humaña AM. 1996. Evolution of rhaphithamnus
venustus (verbenaceae), a gynodioecious hummingbird-pollinated endemic
of the juan fermindez islands, chile. \emph{Pacific Science}.
50(1):55--65

\leavevmode\hypertarget{ref-RN158}{}%
158. Torres-Díaz C, Cavieres LA, Muñoz-Ramírez C, K. Arroyo MT. 2007.
Consecuencias de las variaciones microclimáticas sobre la visita de
insectos polinizadores en dos especies de chaetanthera (asteraceae) en
los andes de chile central. \emph{Revista chilena de historia natural}.
80(4):

\leavevmode\hypertarget{ref-RN159}{}%
159. Torres-Díaz C, Gómez-González S, Stotz GC, Torres-Morales P,
Paredes B, et al. 2011. Extremely long-lived stigmas allow extended
cross-pollination opportunities in a high andean plant. \emph{PLoS ONE}.
6(5):e19497

\leavevmode\hypertarget{ref-RN162}{}%
160. Valdivia CE, Bahamondez A, Simonetti JA. 2011. Negative effects of
forest fragmentation and proximity to edges on pollination and herbivory
of bomarea salsilla (alstroemeriaceae). \emph{Plant Ecology and
Evolution}. 144(3):281--87

\leavevmode\hypertarget{ref-RN81}{}%
161. Valdivia CE, Carroza JP, Orellana JI. 2016. Geographic distribution
and trait-mediated causes of nectar robbing by the european bumblebee
bombus terrestris on the patagonian shrub fuchsia magellanica.
\emph{Flora - Morphology, Distribution, Functional Ecology of Plants}.
225:30--36

\leavevmode\hypertarget{ref-RN79}{}%
162. Valdivia CE, Niemeyer HM. 2006. Do floral syndromes predict
specialisation in plant pollination systems? Assessment of diurnal and
nocturnal pollination of \textless i\textgreater Escallonia
myrtoidea\textless/i\textgreater. \emph{New Zealand Journal of Botany}.
44(2):135--41

\leavevmode\hypertarget{ref-RN160}{}%
163. Valdivia CE, Niemeyer HM. 2006. Do pollinators simultaneously
select for inflorescence size and amount of floral scents? An
experimental assessment on escallonia myrtoidea. \emph{Austral Ecology}.
31(7):897--903

\leavevmode\hypertarget{ref-RN78}{}%
164. Valdivia CE, Niemeyer HM. 2007. Noncorrelated evolution between
herbivore- and pollinator-linked features in aristolochia chilensis
(aristolochiaceae). \emph{Biological Journal of the Linnean Society}.
91(2):239--45

\leavevmode\hypertarget{ref-RN5}{}%
165. Valdivia CE, Orellana JI, Morales-Paredes C. 2018. Ant-mediated
nectar robbing from the chilean firetree embothrium coccineum
(proteaceae): No effect on seed production. \emph{Annales Botanici
Fennici}. 55(4-6):217--26

\leavevmode\hypertarget{ref-RN80}{}%
166. Valdivia CE, Simonetti JA. 2018. The additive effects of
pollinators and herbivores on the vine bomarea salsilla
(alstroemeriaceae), remain spatially consistent in a fragmented forest.
\emph{Revista Mexicana de Biodiversidad}. 89(4):1100--1106

\leavevmode\hypertarget{ref-RN161}{}%
167. Valdivia CE, Simonetti JA, Henríquez CA. 2006. Depressed
pollination of lapageria rosea ruiz et pav. (Philesiaceae) in the
fragmented temperate rainforest of southern south america.
\emph{Biodiversity and Conservation}. 15(5):1845--56

\leavevmode\hypertarget{ref-RN163}{}%
168. Valdovinos FS, Ramos-Jiliberto R, Flores JD, Espinoza C, López G.
2009. Structure and dynamics of pollination networks: The role of alien
plants. \emph{Oikos}. 118(8):1190--1200

\leavevmode\hypertarget{ref-RN82}{}%
169. Vergara RC, Torres-Araneda A, Villagra DA, Raguso RA, Arroyo MTK,
Villagra CA. 2011. Are eavesdroppers multimodal? Sensory exploitation of
flo- ral signals by a non-native cockroach blatta orientalis.
\emph{Current Zoology}. 57(2):162--74

\leavevmode\hypertarget{ref-RN83}{}%
170. Vieli L, Davis FW, Kendall BE, Altieri M. 2016. Landscape effects
on wild bombus terrestris (hymenoptera: Apidae) queens visiting highbush
blueberry fields in south-central chile. \emph{Apidologie}.
47(5):711--16

\leavevmode\hypertarget{ref-RN164}{}%
171. Walter HE. 2010. Floral biology of echinopsis chiloensis ssp.
Chiloensis (cactaceae): Evidence for a mixed pollination syndrome.
\emph{Flora - Morphology, Distribution, Functional Ecology of Plants}.
205(11):757--63

\leavevmode\hypertarget{ref-RN84}{}%
172. Warren SD, Aguilera LE, Baggett LS, Zuñiga M. 2017. Floral
orientation in \textless i\textgreater Eulychnia
acida\textless/i\textgreater{} , an arborescent cactus of the atacama
desert, and implications for cacti globally. \emph{Ecosphere}.
8(10):e01937

\leavevmode\hypertarget{ref-RN85}{}%
173. Willson MF, De Santo TL, Sabag C, Armesto JJ. 1994. Avian
communities of fragmented south-temperate rainforests in chile.
\emph{Conservation Biology}. 8(2):508--20

\end{document}
